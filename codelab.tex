\documentclass{beamer}

\usepackage[utf8]{inputenc}
\usepackage{amssymb}
\usepackage{amsmath}
\usepackage{xcolor}
\usepackage{graphicx}
\usepackage{booktabs}
\usepackage{hyperref}
\usepackage[francais]{babel}

\usetheme{metropolis}
\title{The Deep Learning Codelab}
\date{19 octobre 2017}
\author{Jeff Abrahamson \and Hugo Mougard}
\institute{DevFest Nantes 2017}

% Let me use colours by name.
\newcommand\blue[1]{\textcolor{blue}{#1}}
\newcommand\red[1]{\textcolor{red}{#1}}
\newcommand\green[1]{\textcolor{green}{#1}}
\newcommand\gray[1]{\textcolor{gray}{#1}}
\newcommand\purple[1]{\textcolor{purple}{#1}}
\newcommand\smallgray[1]{\textcolor{gray}{\footnotesize\it #1}}
\newcommand\prevwork[1]{\smallgray{#1}}

% Some ways to include images.
% o=only, f=full, h=fit height.
% more g's or h's means smaller still.
\newcommand\cimgo[1]{\vfill\centerline{\includegraphics{#1}}\vfill}
\newcommand\cimgff[1]{\vfill\centerline{\includegraphics[width=1.2\textwidth]{#1}}\vfill}
\newcommand\cimgf[1]{\vfill\centerline{\includegraphics[width=\textwidth]{#1}}\vfill}
\newcommand\cimg[1]{\vfill\centerline{\includegraphics[width=.9\textwidth]{#1}}\vfill}
\newcommand\cimgg[1]{\vfill\centerline{\includegraphics[width=.8\textwidth]{#1}}\vfill}
\newcommand\cimggg[1]{\vfill\centerline{\includegraphics[width=.7\textwidth]{#1}}\vfill}
\newcommand\cimgsm[1]{\vfill\centerline{\includegraphics[width=.4\textwidth]{#1}}\vfill}
\newcommand\cimgt[1]{\centerline{\includegraphics[width=.2\textwidth]{#1}}\vfill}
\newcommand\cimgh[1]{\vfill\centerline{\includegraphics[height=.9\textheight]{#1}}\vfill}
\newcommand\cimghh[1]{\vfill\centerline{\includegraphics[height=.8\textheight]{#1}}\vfill}
\newcommand\cimghhh[1]{\vfill\centerline{\includegraphics[height=.7\textheight]{#1}}\vfill}
\newcommand\cimghhhh[1]{\vfill\centerline{\includegraphics[height=.6\textheight]{#1}}\vfill}

% Just one phrase centered horizontally and vertically.
\newcommand\vphrase[1]{\vfill\centerline{\large\bf\blue{#1}}\vfill}

\begin{document}
\maketitle

\section{L'apprentissage automatique}
\label{sec:ml}
\begin{frame}{Definition}
  \begin{itemize}
  \item Supervised
  \item Unsupervised
  \item Reinforcement
  \end{itemize}
\end{frame}

\begin{frame}{What is ML?}
  \only<1>{
    \vphrase{Machine learning is not magic}
  }
  \only<2>{
    \vphrase{Machine learning is mathematics}
  }
  \only<3>{
    \vspace{1cm}
    \blue{\bf Mostly, it's these maths:}
    \begin{itemize}
    \item Probability
    \item Statistics
    \item Linear algebra
    \item Optimisation theory
    \item Differential calculus
    \end{itemize}
  }
\end{frame}

\begin{frame}[t]
  \frametitle{What is Statistics}
  
  \begin{enumerate}
  \item<1-3> Identify a question or problem.
  \item<1-3> Collect relevant data on the topic.
  \item<1-3> Analyze the data.
  \item<1-3> Form a conclusion.
  \end{enumerate}
  \only<2>{Sadly, sometimes people forget 1.}
  \only<3>{Statistics is about making 2--4 efficient, rigorous, and meaningful.}
  \only<3>{\vfill\prevwork{\textit{OpenIntro Statistics},
      2nd edition, D.~Diez, C.~Barr, M.~Çetinkaya-Rundel, 2013.}}
\end{frame}

\begin{frame}[t]
  \frametitle{What is data science?}

  \only<1-4>{(Exercise: Is this the same question as the last slide?)}

  \only<1>{
    \begin{enumerate}
    \item Define the question of interest
    \item Get the data
    \item Clean the data
    \item Explore the data
    \item Fit statistical models
    \item Communicate the results
    \item Make your analysis reproducible
    \end{enumerate}
  }
  \only<2>{
    \begin{enumerate}
    \item Define the question of interest
    \item Get the data
    \item Clean the data
    \item \red{Explore the data}
    \item \red{Fit statistical models}
    \item Communicate the results
    \item Make your analysis reproducible
    \end{enumerate}

    \blue{What the public thinks.}
  }
  \only<3>{
    \begin{enumerate}
    \item Define the question of interest
    \item \red{Get the data}
    \item \red{Clean the data}
    \item Explore the data
    \item Fit statistical models
    \item \red{Communicate the results}
    \item \red{Make your analysis reproducible}
    \end{enumerate}

    \blue{Where we spend most of our time.}
  }
  \only<4>{
    \begin{enumerate}
    \item \red{Define the question of interest}
    \item Get the data
    \item Clean the data
    \item Explore the data
    \item Fit statistical models
    \item Communicate the results
    \item Make your analysis reproducible
    \end{enumerate}

    \blue{The easiest part to forget.}
  }
  \only<5>{\vfill\prevwork{\url{http://simplystatistics.org/2015/03/17/} \url{data-science-done-well-looks-easy-and-that-is-a-big-} \url{problem-for-data-scientists/}}}

  \only<6>{\vfill\cimg{images/model-in-one-day.jpg}\vfill}
\end{frame}

\begin{frame}{Representation}
  \vphrase{Typically a vector space}

  \vphrase{Features are dimensions}
\end{frame}

\begin{frame}{Features}
  \vphrase{Feature extraction}

  \vphrase{Feature engineering, synthetic features}
\end{frame}

\begin{frame}{Feature Engineering}
  \begin{enumerate}
  \item Brainstorm
  \item Pick some
  \item Make them
  \item Evaluate
  \item Repeat
  \end{enumerate}
\end{frame}

\section{Les réseaux de neurones artificiels}
\label{sec:nn}
\begin{frame}{Hey I'm a frame}
  Hello, world!
\end{frame}

\section{Les réseaux à convolutions}
\label{sec:convnets}
\begin{frame}{Hey I'm a frame}
  Hello, world!
\end{frame}

\section{Keras}
\label{sec:keras}
\begin{frame}{Start with a plan}
\begin{enumerate}
\item Load Data.
\item Define Model.
\item Compile Model.
\item Fit Model.
\item Evaluate Model.
\item Tie It All Together.
\end{enumerate}
\end{frame}

\section{MNIST}
\label{sec:mnist}
\begin{frame}{Hey I'm a frame}
  Hello, world!
\end{frame}

\section{Infra}
\label{sec:infra}
\begin{frame}{Hey I'm a frame}
  Hello, world!
\end{frame}

\section{Modèle}
\label{sec:model}
\begin{frame}{Hey I'm a frame}
  Hello, world!
\end{frame}

\section{Ressources}
\label{sec:resources}
\begin{frame}{Hey I'm a frame}
  Hello, world!
\end{frame}

\section{Q \& A}
\label{sec:qa}
\end{document}